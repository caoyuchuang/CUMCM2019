\documentclass{ctexart}
% ctexart 自带的边距太大了费纸
% 格式要求里写最好不超过 20 页
\usepackage[hmargin=3cm,vmargin=4cm]{geometry}
\usepackage{xcolor}
\usepackage{booktabs}
\usepackage{amsmath}
\usepackage{lmodern}
\setmainfont{Times New Roman}
\begin{document}
% 取消 ctexart 的页眉标题,并把页码移到页脚
\pagestyle{plain}
\title{标题}
% 不能添加作者,最后输出文档时应该把下一行注释掉,写在这里是防 warning
\author{}
% 不需要日期,不过如果不这么设置 ctexart 会自动加上 today
\date{}
\maketitle
\section*{摘要}

本文不包含前两页(承诺书 + 评审表格),这是因为每年前两页的内容都可能有变化;而且在所提供的 Word 上填写比设计 \LaTeX 样式容易得多\cite{someitem}。为此也将本文中的字体从 \verb|ctexart| 默认的 Computer Modern Roman 改成了 Times New Roman。

摘要第二段。

\paragraph*{关键词} 关键词一\quad 关键词二
\newpage
\section{问题的重述}
\subsection{引言}
问题的背景。
\subsection{问题的提出}
用自己的话将问题重述一遍,描述清楚问题涉及的范围,输入信息和输出信息。
\newpage
\section{问题的分析}
\subsection{问题一的分析}
\subsection{问题二的分析}
\subsection{问题三的分析}
\newpage
\section{模型的建立}
\subsection{模型的假设}
进行合理近似,说明能够作出近似的原因,用数据说明这样做可能引起的误差有多大。
\subsection{(模型主体部分)}
\subsection{符号说明}
用表格的形式说明不同符号对应着什么物理量。表格最好使用三线表,不过手敲 LaTeX 表格代码太慢,可以使用这个网站:\verb|http://www.tablesgenerator.com/latex_tables|,在图形界面上输入表格之后,选择 Booktabs Style,点击 Generate,再 Copy to Clipboard 就可以了。如果表格中有公式(表格里肯定有公式,因为物理量都是斜体),要在输入的时候就把 \$ 符号带上。

\begin{table}[h]
\centering
\caption{注释}
\begin{tabular}{cc}
\toprule
物理量 & 符号  \\ \midrule
位置  & $q$ \\
动量  & $p$ \\
……  & ……  \\ \bottomrule
\end{tabular}
\end{table}
\newpage
\section{模型的求解}
必要的推导和结果展示。

块公式可以统一标号,也可以统一不标号。不标号公式:

$$
y=x^2
$$

标号公式:

\begin{equation}
y=x^2
\end{equation}

多行公式最好用 \verb|aligned| 环境。

$$
\begin{aligned}
a&=b\\
&=c\\
&=d
\end{aligned}
$$

多行公式标号:

\begin{equation}
\begin{aligned}
a&=b\\
&=c\\
&=d
\end{aligned}
\end{equation}

图使用 \verb|graphicx| 宏包。考虑到大部分图是数据作出来的图,可以考虑在 \verb|matplotlib.pyplot| 的基础上进行进一步的配置以增强美观性和一致性,如字体(改为衬线字体),颜色,边距等等,然后封装成一个 \verb|class|。
\subsection{问题一的求解}
\subsection{问题二的求解}
\subsection{问题三的求解}
\newpage
\section{模型的评价}
优点、缺点、潜在改进空间和应用范围。
\newpage
\begin{thebibliography}{100}
\bibitem{someitem} 作者.题目[J].期刊,年份,卷(期):起始页-终止页.
\end{thebibliography}
\newpage
\appendix
\section{源代码}
用什么写可以再讨论,现在以 Python 为例说明一下编写和提交的一些规范性问题。
\subsection{说明}
这里应该注明一些东西:
\begin{itemize}
    \item 语言版本(如 Python 3.7.4)
    \item 编译运行环境(越通用越好,比如 Python 最好在自己平时用的集成式开发环境中开发完成后放到自带的 IDLE 中看看能否运行)
    \item 各个源代码文件的输入、输出、调用关系等等,对于这种 1000 行代码左右的中小型项目来说,比较好的开发方式是一个模块文件 + 几个小题分别调用模块。以 Python 为例,比较好的做法是在 \verb|lib.py| 中封装一个 \verb|class|,然后每个小题 \verb|import lib| 完成问题的求解和输入输出等等。
    \item 如果运行时间比较长要说明
\end{itemize}
\subsection{模块源代码}
\subsection{第一题源代码}
这里给出一个代码(含高亮)的示例。
\lstinputlisting[
    style       =   Python,
    caption     =   {源代码.py},
    label       =   {源代码.py}
]{源代码.py}
\subsection{第二题源代码}
\subsection{第三题源代码}
\newpage
\section{(其他查阅使用的数据资料)}
\end{document}